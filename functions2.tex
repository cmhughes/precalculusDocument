\chapter{Functions}
\minitoc

\section{Function algebra}
%===================================
%   Author: Hughes
%   Date:   October 2012
%===================================
\begin{pccdefinition}[Function algebra]
Given two functions $f$ and $g$, we may combine the two functions to 
form new functions
\[
	f+g, \qquad f-g, \qquad f\cdot g, \qquad \frac{f}{g}
\]
The formula for each function can be found using
\begin{align*}
	(f+g)(x)                      & =f(x)+g(x)         \\ 
	(f-g)(x)                      & =f(x)-g(x)         \\ 
	(f\cdot g)(x)                 & =f(x)\cdot g(x)    \\ 
	\left( \frac{f}{g} \right)(x) & =\frac{f(x)}{g(x)} 
\end{align*}
The domain of each of the functions $f+g$, $f-g$, and $f\cdot g$ is
\[
	(\text{domain of }f)\cap (\text{domain of }g)
\]
The domain of the function $\frac{f}{g}$ is
\[
	(\text{domain of }f)\cap (\text{domain of }g \text{ such that }g(x)\ne 0)
\]
\end{pccdefinition}
%===================================
%   Author: Hughes
%   Date:   October 2012
%===================================
\begin{pccexample}[Function algebra domain]
In each of the following cases you are given the formulas for two 
functions $f$ and $g$. In each case, find the domain of $f+g$
and $\frac{f}{g}$.
\begin{enumerate}
	\item $f(x)=\sqrt{x}$, $g(x)=\sqrt{1-x}$
	\item $f(x)=\sqrt{x-1}$, $g(x)=\sqrt{1-x}$
	\item $f(x)=\dfrac{1}{x+3}$, $g(x)=\sqrt{x+5}$
\end{enumerate}
\begin{pccsolution}
\begin{enumerate}
	\item The domain of $f$ is $[0,\infty)$ and the domain of $g$ is 
	$(-\infty,1]$.  Therefore, the domain of the function $f+g$ is
	\[
		[0,\infty)\cap (-\infty,1]=[0,1]
	\]
	The domain of the function $\frac{f}{g}$ is found in a similar 
	way, except we must have the additional condition that $g(x)\ne 0$;
	we therefore must exclude $1$ from the domain. The domain 
	of $\frac{f}{g}$ is therefore
	\[
		[0,1)
	\]
	\item The domain of $f$ is $[1,\infty)$ and the domain of $g$ is
	$(-\infty,1]$. Therefore the domain of $f+g$ is 
	\[
		[1,\infty)\cap (-\infty,1]= \{ 1\}
	\]
	The domain of the function $\frac{f}{g}$ is found in a simlar way, but
	we must exclude all values of $x$ that make $g(x)=0$. Since $g(1)=0$
	we must exclude $1$ from the domain of $\frac{f}{g}$; we therefore 
	conclude that the domain of $\frac{f}{g}$ is the empty set, $\emptyset$.
	\item The domain of $f$ is $(-\infty,-3)\cup (-3,\infty)$ and the 
	domain of $g$ is $[-5,\infty]$. The domain of $f+g$ is therefore
	\[
		((-\infty,-3)\cup (-3,\infty))\cap [-5,\infty ) = [-5,-3)\cup (-3,\infty)
	\]
	We must exclude $-5$ from the domain of $\frac{f}{g}$ since $g(-5)=0$; 
	the domain of $\frac{f}{g}$ is
	\[
		(-5,-3)\cup (-3,\infty)
	\]
\end{enumerate}
\end{pccsolution}
\end{pccexample}

\begin{exercises}
%===================================
%   Author: Hughes
%   Date:   October 2012
%===================================
\begin{problem}[Function algebra using formulas]
In each of the following problems you are given formulas for 
functions $f$ and $g$. Find the domain of $f\cdot g$ and 
$\frac{f}{g}$ in each case.
\begin{multicols}{2}
	\begin{subproblem}
		$f(x)=x$, $g(x)=x^2+1$ 
		\begin{shortsolution}
			Domain of $f\cdot g$: $(-\infty,\infty)$; domain of $\frac{f}{g}$: $(-\infty,\infty)$.  
		\end{shortsolution}
	\end{subproblem}
	\begin{subproblem}
		$f(x)=3x+2$, $g(x)=\sqrt{x}$  
		\begin{shortsolution}
			Domain of $f\cdot g$: $[0,\infty)$; domain of $\frac{f}{g}$: $(0,\infty)$.
		\end{shortsolution}
	\end{subproblem}
	\begin{subproblem}
		$f(x)=\sqrt[4]{x-1}$, $g(x)=x^2+5x+4$  
		\begin{shortsolution}
			Domain of $f\cdot g$: $[1,\infty)$; domain of $\frac{f}{g}$: $[1,\infty)$  
		\end{shortsolution}
	\end{subproblem}
	\begin{subproblem}
		$f(x)=\sqrt[5]{x}$, $g(x)=x^2-9x-10$  
		\begin{shortsolution}
			Domain of $f\cdot g$: $(-\infty,\infty)$; domain of $\frac{f}{g}$: $(-\infty,-1)\cup (-1,10)\cup (10,\infty)$. 
		\end{shortsolution}
	\end{subproblem}
\end{multicols}
\end{problem}
%===================================
%   Author: Hughes
%   Date:   October 2012
%===================================
\begin{problem}[Function algebra numerically]\label{fun:prob:combine}
Values of the functions $f$, $g$, $h$, and $j$ are shown in 
\crefrange{fun:tab:combinef}{fun:tab:combinej}

\begin{table}[!htb]
	\centering
	\begin{widepage}
	\caption{Tables for \cref{fun:prob:combine}}
	\label{fun:tab:combine}
	\begin{subtable}{.2\textwidth}
		\centering
		\caption{$y=f(x)$}
		\label{fun:tab:combinef}
		\begin{tabular}{S[table-format=1.0]S[table-format=2.0]}
			\beforeheading
			\heading{$x$} & \heading{$y$} \\            
			\afterheading
			-4          & -56         \\\normalline 
			-3          & -18         \\\normalline  
			-2          & 0           \\\normalline   
			-1          & 4           \\\normalline  
			0           & 0           \\\normalline   
			1           & -6          \\\normalline   
			2           & -8          \\\normalline  
			3           & 0           \\\normalline  
			4           & 24          \\\lastline    
		\end{tabular}
	\end{subtable}
	\hfill
	\begin{subtable}{.2\textwidth}
		\centering
		\caption{$y=g(x)$}
		\label{fun:tab:combineg}
		\begin{tabular}{S[table-format=1.0]S[table-format=3.0]}
			\beforeheading
			\heading{$x$} & \heading{$y$} \\ \afterheading 
			-4          & -16         \\\normalline       
			-3          & -3          \\\normalline        
			-2          & 0           \\\normalline         
			-1          & -1          \\\normalline        
			0           & 0           \\\normalline         
			1           & 9           \\\normalline         
			2           & 32          \\\normalline        
			3           & 75          \\\normalline        
			4           & 144         \\\lastline          
		\end{tabular}
	\end{subtable}
	\hfill
	\begin{subtable}{.2\textwidth}
		\centering
		\caption{$y=h(x)$}
		\label{fun:tab:combineh}
		\begin{tabular}{S[table-format=1.0]S[table-format=2.0]}
			\beforeheading
			\heading{$x$} & \heading{$y$} \\ \afterheading 
			-4          & 2           \\\normalline       
			-3          & 4           \\\normalline         
			-2          & 6           \\\normalline       
			-1          & 8           \\\normalline         
			0           & 10          \\\normalline         
			1           & 12          \\\normalline         
			2           & 14          \\\normalline       
			3           & 16          \\\normalline         
			4           & 18          \\\lastline          
		\end{tabular}
	\end{subtable}
	\hfill
	\begin{subtable}{.2\textwidth}
		\centering
		\caption{$y=j(x)$}
		\label{fun:tab:combinej}
		\begin{tabular}{S[table-format=1.0]S[table-format=3.0]}
			\beforeheading
			\heading{$x$} & \heading{$y$} \\ \afterheading 
			-4          & 30          \\\normalline        
			-3          & 21          \\\normalline         
			-2          & 12          \\\normalline        
			-1          & 3           \\\normalline         
			0           & -6          \\\normalline         
			1           & -15         \\\normalline         
			2           & 15          \\\normalline        
			3           & 96          \\\normalline        
			4           & 760         \\\lastline          
		\end{tabular}
	\end{subtable}
	\end{widepage}
\end{table}

Construct a table of values for each of the following functions, marking
with an X any that undefined.
\begin{multicols}{6}
	\begin{subproblem}
		$f+g$
		\begin{shortsolution}
		\begin{tabular}[t]{S[table-format=1.0]S[table-format=3.0]}
				\beforeheading
				\heading{$x$} & \heading{$(f+g)(x)$} \\ \afterheading 
				-4          & -72                \\\normalline        
				-3          & -21                \\\normalline         
				-2          & 0                  \\\normalline        
				-1          & 3                  \\\normalline         
				0           & 0                  \\\normalline         
				1           & 3                  \\\normalline         
				2           & 24                 \\\normalline        
				3           & 75                 \\\normalline        
				4           & 168                \\\lastline          
			\end{tabular}
		\end{shortsolution}
	\end{subproblem}
	\begin{subproblem}
		$f-g$  
		\begin{shortsolution}
		\begin{tabular}[t]{S[table-format=1.0]S[table-format=3.0]}
				\beforeheading
				\heading{$x$} & \heading{$(f-g)(x)$} \\ \afterheading 
				-4          & -40                \\\normalline        
				-3          & -15                \\\normalline         
				-2          & 0                  \\\normalline        
				-1          & 5                  \\\normalline         
				0           & 0                  \\\normalline         
				1           & -15                \\\normalline         
				2           & -40                \\\normalline        
				3           & -75                \\\normalline        
				4           & -120               \\\lastline          
			\end{tabular}
		\end{shortsolution}
	\end{subproblem}
	\begin{subproblem}
		$g\cdot h$  
		\begin{shortsolution}
		\begin{tabular}[t]{S[table-format=1.0]S[table-format=4.0]}
				\beforeheading
				\heading{$x$} & \heading{$(g\cdot h)(x)$} \\ \afterheading 
				-4          & -32                     \\\normalline        
				-3          & -12                     \\\normalline         
				-2          & 0                       \\\normalline        
				-1          & -8                      \\\normalline         
				0           & 0                       \\\normalline         
				1           & 108                     \\\normalline         
				2           & 448                     \\\normalline        
				3           & 1200                    \\\normalline        
				4           & 2592                    \\\lastline          
			\end{tabular}
		\end{shortsolution}
	\end{subproblem}
	\begin{subproblem}
		$h+j$  
		\begin{shortsolution}
		\begin{tabular}[t]{S[table-format=1.0]S[table-format=3.0]}
				\beforeheading
				\heading{$x$} & \heading{$(h+j)(x)$} \\ \afterheading 
				-4          & 32                 \\\normalline        
				-3          & 25                 \\\normalline         
				-2          & 18                 \\\normalline        
				-1          & 11                 \\\normalline         
				0           & 4                  \\\normalline         
				1           & -3                 \\\normalline         
				2           & 29                 \\\normalline        
				3           & 112                \\\normalline        
				4           & 778                \\\lastline          
			\end{tabular}
		\end{shortsolution}
	\end{subproblem}
	\begin{subproblem}
		$\left( \frac{j}{h} \right)$  
		\begin{shortsolution}
		\begin{tabular}[t]{S[table-format=1.0]S[table-format=2.0]}
				\beforeheading
				\heading{$x$} & \heading{$\left( \frac{j}{h} \right)(x)$} \\ \afterheading 
				-4          & 15                                      \\\normalline        
				-3          & \num{21/4}                            \\\normalline         
				-2          & 2                                       \\\normalline        
				-1          & \num{3/8}                             \\\normalline         
				0           & \num{-3/5}                            \\\normalline         
				1           & \num{-5/4}                            \\\normalline         
				2           & \num{15/14}                           \\\normalline        
				3           & 6                                       \\\normalline        
				4           & \num{380/9}                           \\\lastline          
			\end{tabular}
		\end{shortsolution}
	\end{subproblem}
	\begin{subproblem}
		$\left( \frac{j}{f} \right)$  
		\begin{shortsolution}
		\begin{tabular}[t]{S[table-format=1.0]S[table-format=1.0]}
				\beforeheading
				\heading{$x$} & \heading{$\left( \frac{j}{f} \right)(x)$} \\ \afterheading 
				-4          & \num{-15/28}                          \\\normalline        
				-3          & \num{-7/6}                            \\\normalline         
				-2          & X                                         \\\normalline        
				-1          & \num{3/4}                             \\\normalline         
				0           & X                                         \\\normalline         
				1           & \num{5/2}                             \\\normalline         
				2           & \num{-15/8}                           \\\normalline        
				3           & X                                         \\\normalline        
				4           & \num{95/3}                            \\\lastline          
			\end{tabular}
		\end{shortsolution}
	\end{subproblem}
\end{multicols}
\end{problem}



%===================================
%   Author: Hughes
%   Date:   October 2012
%===================================
\begin{problem}[Function algebra graphically]
Consider the functions $F$, $G$, $H$, and $J$ that have been graped in 
\cref{fun:fig:algebra}. Use the graphs to plot each of the following 
functions.
\begin{multicols}{4}
	\begin{subproblem}
		$F+G$ 
		\begin{shortsolution}
			The function $F+G$ is shown below.
			
			\begin{tikzpicture}
				\begin{axis}[
					framed,
					xmin=-3,xmax=3,
					ymin=-4,ymax=3,
					xtick={-2,0,...,2},
					ytick={-2,0,...,2},
					minor xtick={-1,1},
					minor ytick={-3,-1,1},
					grid=both,
					]
					\addplot[pccplot,-] expression[domain=-2:-1]{0};
					\addplot[pccplot,-] expression[domain=-1:0]{0};
					\addplot[pccplot,-] expression[domain=0:1]{0};
					\addplot[pccplot,-] expression[domain=1:2]{-3};
					\addplot[holdot]coordinates{(-2,0)(-1,0)(0,0)(1,0)(1,-3)(2,-3)};
				\end{axis}
			\end{tikzpicture}
		\end{shortsolution}
	\end{subproblem}
	\begin{subproblem}
		$G\cdot H$  
		\begin{shortsolution}
			The function $G\cdot H$ is shown below.
			
			\begin{tikzpicture}
				\begin{axis}[
					framed,
					xmin=-3,xmax=3,
					ymin=-3,ymax=3,
					xtick={-2,0,...,2},
					ytick={-2,0,...,2},
					minor xtick={-1,1},
					minor ytick={-1,1},
					grid=both,
					]
					\addplot[pccplot,-] expression[domain=-2:-1]{1};
					\addplot[pccplot,-] expression[domain=-1:0]{0};
					\addplot[pccplot,-] expression[domain=0:1]{1};
					\addplot[pccplot,-] expression[domain=1:2]{-2};
					\addplot[holdot]coordinates{(-2,1)(-1,1)(-1,0)(0,0)(0,1)(1,1)(1,-2)(2,-2)};
				\end{axis}
			\end{tikzpicture}
		\end{shortsolution}
	\end{subproblem}
	\begin{subproblem}
		$\frac{H}{J}$  
		\begin{shortsolution}
			The function $\frac{H}{J}$ is shown below; note that this function is
			undefined on the interval $(0,1)$.
			
			\begin{tikzpicture}
				\begin{axis}[
					framed,
					xmin=-3,xmax=3,
					ymin=-3,ymax=3,
					xtick={-2,0,...,2},
					ytick={-2,0,...,2},
					minor xtick={-1,1},
					minor ytick={-1,1},
					grid=both,
					]
					\addplot[pccplot,-] expression[domain=-2:-1]{-.5};
					\addplot[pccplot,-] expression[domain=-1:0]{0};
					\addplot[pccplot,-] expression[domain=1:2]{-1};
					\addplot[holdot]coordinates{(-2,-0.5)(-1,-0.5)(-1,0)(0,0)(1,-1)(2,-1)};
				\end{axis}
			\end{tikzpicture}
		\end{shortsolution}
	\end{subproblem}
	\begin{subproblem}
		$J-F$  
		\begin{shortsolution}
			The function $J-F$ is shown below.
			
			\begin{tikzpicture}
				\begin{axis}[
					framed,
					xmin=-3,xmax=3,
					ymin=-3,ymax=3,
					xtick={-2,0,...,2},
					ytick={-2,0,...,2},
					minor xtick={-1,1},
					minor ytick={-1,1},
					grid=both,
					]
					\addplot[pccplot,-] expression[domain=-2:-1]{1};
					\addplot[pccplot,-] expression[domain=-1:0]{-1};
					\addplot[pccplot,-] expression[domain=0:1]{1};
					\addplot[pccplot,-] expression[domain=1:2]{-1};
					\addplot[holdot]coordinates{(-2,1)(-1,1)(-1,-1)(0,-1)(0,1)(1,1)(1,-1)(2,-1)};
				\end{axis}
			\end{tikzpicture}
		\end{shortsolution}
	\end{subproblem}
\end{multicols}

\begin{figure}[!htb]
	\begin{widepage}
	\setlength{\figurewidth}{.2\textwidth}
	\centering
	\begin{subfigure}{\figurewidth}
		\begin{tikzpicture}
			\begin{axis}[
				framed,
				xmin=-3,xmax=3,
				ymin=-3,ymax=3,
				xtick={-2,0,...,2},
				ytick={-2,0,...,2},
				minor xtick={-1,1},
				minor ytick={-1,1},
				grid=both,
				]
				\addplot[pccplot,-] expression[domain=-2:-1]{1};
				\addplot[pccplot,-] expression[domain=-1:0]{2};
				\addplot[pccplot,-] expression[domain=0:1]{-1};
				\addplot[pccplot,-] expression[domain=1:2]{-2};
				\addplot[holdot]coordinates{(-2,1)(-1,1)(-1,2)(0,2)(0,-1)(1,-1)(1,-2)(2,-2)};
			\end{axis}
		\end{tikzpicture}
		\caption{$y=F(x)$}
		\label{fun:fig:algebra1}
	\end{subfigure}
	\hfill
	\begin{subfigure}{\figurewidth}
		\begin{tikzpicture}
			\begin{axis}[
				framed,
				xmin=-3,xmax=3,
				ymin=-3,ymax=3,
				xtick={-2,0,...,2},
				ytick={-2,0,...,2},
				minor xtick={-1,1},
				minor ytick={-1,1},
				grid=both,
				]
				\addplot[pccplot,-] expression[domain=-2:-1]{-1};
				\addplot[pccplot,-] expression[domain=-1:0]{-2};
				\addplot[pccplot,-] expression[domain=0:1]{1};
				\addplot[pccplot,-] expression[domain=1:2]{-1};
				\addplot[holdot]coordinates{(-2,-1)(-1,-1)(-1,-2)(0,-2)(0,1)(1,1)(1,-1)(2,-1)};
			\end{axis}
		\end{tikzpicture}
		\caption{$y=G(x)$}
		\label{fun:fig:algebra2}
	\end{subfigure}
	\hfill
	\begin{subfigure}{\figurewidth}
		\begin{tikzpicture}
			\begin{axis}[
				framed,
				xmin=-3,xmax=3,
				ymin=-3,ymax=3,
				xtick={-2,0,...,2},
				ytick={-2,0,...,2},
				minor xtick={-1,1},
				minor ytick={-1,1},
				grid=both,
				]
				\addplot[pccplot,-] expression[domain=-2:-1]{-1};
				\addplot[pccplot,-] expression[domain=-1:0]{0};
				\addplot[pccplot,-] expression[domain=0:1]{1};
				\addplot[pccplot,-] expression[domain=1:2]{2};
				\addplot[holdot]coordinates{(-2,-1)(-1,0)(-1,-1)(0,0)(0,1)(1,1)(1,2)(2,2)};
			\end{axis}
		\end{tikzpicture}
		\caption{$y=H(x)$}
		\label{fun:fig:algebra3}
	\end{subfigure}
	\hfill
	\begin{subfigure}{\figurewidth}
		\begin{tikzpicture}
			\begin{axis}[
				framed,
				xmin=-3,xmax=3,
				ymin=-3,ymax=3,
				xtick={-2,0,...,2},
				ytick={-2,0,...,2},
				minor xtick={-1,1},
				minor ytick={-1,1},
				grid=both,
				]
				\addplot[pccplot,-] expression[domain=-2:-1]{2};
				\addplot[pccplot,-] expression[domain=-1:0]{1};
				\addplot[pccplot,-] expression[domain=0:1]{0};
				\addplot[pccplot,-] expression[domain=1:2]{-2};
				\addplot[holdot]coordinates{(-2,2)(-1,2)(-1,1)(0,1)(0,0)(1,0)(1,-2)(2,-2)};
			\end{axis}
		\end{tikzpicture}
		\caption{$y=J(x)$}
		\label{fun:fig:algebra4}
	\end{subfigure}
	\caption{}
	\label{fun:fig:algebra}
	\end{widepage}
\end{figure}
\end{problem}


%===================================
%   Author: Hughes
%   Date:   October 2012
%===================================
\begin{problem}[Function algebra numerically]
\Cref{fun:tab:algebranum} shows some values of the functions $f$, $g$, 
and some functions obtained by using some function algebra on $f$ and $g$. 
Use the given values to complete \cref{fun:tab:algebranum}.
\begin{shortsolution}
	\begin{tabular}[t]{cS[table-format=2.0]*{4}S[table-format=1.0]S[parse-numbers=false]S[table-format=2.0]}
		\beforeheading
		$x$                             & -6          & -4 & -2 & 0  & 2  & 4     & 6  \\\normalline
		$f(x)$                          & 2           & 1  & 3  & 0  & 2  & \pi   & 12 \\\normalline
		$g(x)$                          & 8           & 1  & 3  & 5  & -1 & -1    & 2  \\
		\afterheading
		$(f+g)(x)$                      & 10          & 2  & 6  & 5  & 1  & \pi-1 & 14 \\\normalline 
		$(f-g)(x)$                      & -6          & 0  & 0  & -5 & 3  & \pi+1 & 10 \\\normalline
		$(f\cdot g)(x)$                 & 16          & 1  & 9  & 0  & -2 & -\pi  & 24 \\\normalline
		$\left( \frac{f}{g} \right)(x)$ & \num{1/4} & 1  & 1  & 0  & -2 & -\pi  & 6  \\\lastline
	\end{tabular}
\end{shortsolution}

\begin{table}[htb]
	\centering
	\caption{}
	\label{fun:tab:algebranum}
	\begin{tabular}{c*{7}S[table-format=1.0]}
		\beforeheading
		$x$                             & -6 & -4 & -2 & 0 & 2 & 4   & 6  \\\normalline
		$f(x)$                          & 2  &      & 3  &     &     & \pi &      \\\normalline
		$g(x)$                          & 8  & 1  &      & 5 &     & -1  &      \\
		\afterheading
		$(f+g)(x)$                      &      & 2  &      &     & 1 &       &      \\\normalline 
		$(f-g)(x)$                      &      &      &      &     & 3 &       & 10 \\\normalline
		$(f\cdot g)(x)$                 &      &      &      & 0 &     &       &      \\\normalline
		$\left( \frac{f}{g} \right)(x)$ &      &      & 1  &     &     &       & 6  \\\lastline
	\end{tabular}
\end{table}
\end{problem}

%===================================
%   Author: Hughes
%   Date:   October 2012
%===================================
\begin{problem}[Function algebra graphically]
Consider the functions $\alpha$, $\beta$, $\gamma$, and $\delta$ which 
have been graphed in \cref{fun:fig:algvarious}. 
Evaluate each of the following.
\begin{multicols}{4}
	\begin{subproblem}
		$(\alpha+\beta)(0)$
		\begin{shortsolution}
			$-2$ 
		\end{shortsolution}
	\end{subproblem}
	\begin{subproblem}
		$(\beta-\gamma)(3)$ 
		\begin{shortsolution}
			$5$ 
		\end{shortsolution}
	\end{subproblem}
	\begin{subproblem}
		$(\gamma\cdot\delta)(2)$ 
		\begin{shortsolution}
			$-2$ 
		\end{shortsolution}
	\end{subproblem}
	\begin{subproblem}
		$\left( \frac{\delta}{\alpha} \right)(0)$ 
		\begin{shortsolution}
			$\frac{1}{2}$ 
		\end{shortsolution}
	\end{subproblem}
\end{multicols}

\begin{figure}[!htb]
	\begin{widepage}
	\setlength{\figurewidth}{.2\textwidth}
	\centering
	\begin{subfigure}{\figurewidth}
		\begin{tikzpicture}
			\begin{axis}[
				framed,
				xmin=-5,xmax=5,
				ymin=-5,ymax=5,
				xtick={-4,-2,...,4},
				ytick={-4,-2,...,4},
				minor xtick={-3,-1,...,3},
				minor ytick={-3,-1,...,3},
				grid=both,
				]
				\addplot expression[domain=-3:3]{(x+2)*(x-2)};
			\end{axis}
		\end{tikzpicture}
		\caption{$y=\alpha(x)$}
	\end{subfigure}
	\hfill
	\begin{subfigure}{\figurewidth}
		\begin{tikzpicture}
			\begin{axis}[
				framed,
				xmin=-5,xmax=5,
				ymin=-5,ymax=5,
				xtick={-4,-2,...,4},
				ytick={-4,-2,...,4},
				minor xtick={-3,-1,...,3},
				minor ytick={-3,-1,...,3},
				grid=both,
				]
				\addplot expression[domain=-5:5]{2};
			\end{axis}
		\end{tikzpicture}
		\caption{$y=\beta(x)$}
	\end{subfigure}
	\hfill
	\begin{subfigure}{\figurewidth}
		\begin{tikzpicture}
			\begin{axis}[
				framed,
				xmin=-5,xmax=5,
				ymin=-5,ymax=5,
				xtick={-4,-2,...,4},
				ytick={-4,-2,...,4},
				minor xtick={-3,-1,...,3},
				minor ytick={-3,-1,...,3},
				grid=both,
				]
				\addplot expression[domain=-5:5]{-x};
			\end{axis}
		\end{tikzpicture}
		\caption{$y=\gamma(x)$}
	\end{subfigure}
	\hfill
	\begin{subfigure}{\figurewidth}
		\begin{tikzpicture}
			\begin{axis}[
				framed,
				xmin=-5,xmax=5,
				ymin=-5,ymax=5,
				xtick={-4,-2,...,4},
				ytick={-4,-2,...,4},
				minor xtick={-3,-1,...,3},
				minor ytick={-3,-1,...,3},
				grid=both,
				]
				\addplot expression[domain=-5:2.7]{2^x-3};
			\end{axis}
		\end{tikzpicture}
		\caption{$y=\delta(x)$}
	\end{subfigure}
	\caption{}
	\label{fun:fig:algvarious}
	\end{widepage}
\end{figure}
\end{problem}
\end{exercises}


\section{Piecewise-defined functions}
The functions that we have considered so far have had just one formula 
throughout their domain; for example, the quadratic function $q$ that 
has formula
\[
    q(x)=5-3x^2
\]
is defined for all real numbers.
\begin{marginfigure}
  \centering
\begin{tikzpicture}
  \begin{axis}[
    framed,
    xmin=-1,xmax=5,
    ymin=-1,ymax=2,
    xtick={1,2,...,4},
    ytick={1},
    xlabel={$t$},
    grid=major,
    ]
    \addplot+[-]expression[domain=0:1]{0};
    \addplot[pccplot,-]expression[domain=1:2]{1};
    \addplot[pccplot,-]expression[domain=2:3]{0};
    \addplot[pccplot,-]expression[domain=3:4]{1};
    \addplot[holdot]coordinates{(1,0)(2,1)(3,0)(4,1)};
    \addplot[soldot]coordinates{(0,0)(1,1)(2,0)(3,1)};
  \end{axis}
\end{tikzpicture}
\captionof{figure}{A switch function}
\label{fun:fig:electric}
\end{marginfigure}

There are many applications for which this is 
too restrictive; for example, electrical engineers often work with switches 
that are turned on (with a value of $1$) and off (with a value of $0$). 
An example of a function that might model such a switch over time, $t$, is shown 
in \cref{fun:fig:electric}.
It is clear that this function takes the value $0$ on some intervals, 
and $1$ on other intervals. We can write a formula for such a function
by first noting that is a \emph{piecewise-defined} function.


\begin{pccdefinition}[Piecewise-defined functions]
  A piecewise-defined function has different formulas for different parts
  of its domain. 

  The formula for a piecewise-defined function is written using a \emph{left brace}
  $\{$ and is read from top to bottom as we move from left to right 
  through its domain.
\end{pccdefinition}

%===================================
%   Author: Hughes
%   Date:   November 2012
%===================================
\begin{pccexample}
Find a formula for the function that is graphed in \cref{fun:fig:electric}. 
\begin{pccsolution}
  Let's assume that the function shown in \cref{fun:fig:electric} is called $f$. 
  It seems that $f(t)$ takes the value $0$ on the intervals $[0,1)$ and $[2,3)$; 
  similarly, $f(t)$ takes the value $1$ on the intervals $[1,2)$ and $[3,4)$.
  We can translate this into a formula for the function $f$ as follows
  \[
        f(t)=
        \begin{cases}
          0,& 0\leq t <1\\
          1,& 1\leq t<2\\
          0,& 2\leq t<3\\
          1,& 3\leq t<4
        \end{cases}
  \]
  Note that we use the left brace, $\{$, to link the formula together. Note 
  also that the domain of $f$ is $[0,4)$ and that as we read the formula from top
  to bottom, the values of $t$ go from left to right. This will be true in 
  every piecewise-defined function that we see.
\end{pccsolution}
\end{pccexample}

%===================================
%   Author: Hughes/Fresh/Barkin
%   Date:   November 2012
%===================================
\begin{pccexample}[Coupons] 
  \pccname{Wendy} is going shopping at \pccname{Jessica}'s
  beauty salon. Wendy has the coupons shown in \cref{fun:fig:coupons}. 
  Wendy is very interested in modeling the total amount of money that 
  she will spend after applying the discounts from the coupons. 

\begin{figure}[!htb]
\begin{widepage}
  \centering
\begin{subfigure}{.4\textwidth}
\resizebox{\textwidth}{!}{\begin{tikzpicture}
\node (reduction) at (-1,0)[scale=7] {\$5};
\node [right=of reduction](description)[scale=2]{\$5 off any purchase less than \$15};
\begin{pgfonlayer}{background}
\node (jessica) at (5,0.25)[scale=5,text=black!20] {\begin{tabular}{c}Jessica's\\ \emph{beauty}\\ salon\end{tabular}};
\node [fit=(reduction)(description)(jessica),
				draw=blue,thick,
				rounded corners,fill=red!15]{};
\end{pgfonlayer}
\end{tikzpicture}}
\caption{}
\end{subfigure}%
\hfill
\begin{subfigure}{.4\textwidth}
\resizebox{\textwidth}{!}{\begin{tikzpicture}
\node (reduction) at (-1,0)[scale=7] {20\%};
\node [right=of reduction](description)[scale=2]{20\% off any purchase \$15 or more};
\begin{pgfonlayer}{background}
\node [fit=(reduction)(description)(jessica),
				draw=blue,thick,
				rounded corners,fill=purple!35]{};
\node (jessica) at (5,0.25)[scale=5,text=black!20] {\begin{tabular}{c}Jessica's\\ 
\emph{beauty}\\salon \end{tabular}};
\end{pgfonlayer}
\end{tikzpicture}}
\caption{}
\end{subfigure}
\caption{Wendy's coupons}
\label{fun:fig:coupons}
\end{widepage}
\end{figure}

  Wendy observes that the amount of money that she will save depends 
  on the total cost of the items. She decides to let the function $d$ 
  represent the cost of the items after applying the discount to 
  items that cost $x$ dollars initially. Wendy realizes that she needs one
   formula for items that cost below \$15, and one for items 
   that cost \$15 or more; she decides to write a formula for $d(x)$ 
   using a piecewise-defined formula 
\[
    d(x)=
    \begin{cases}
      x-5, & 0<x<15\\
      0.8x, & x\geq 15
    \end{cases}
\]
Wendy decides to test her formula by finding how much an item that 
costs \$13 initially will cost after using the coupon. She evaluates $d(13)$
\begin{align*}
  d(13)&=13-5\\
  &=8
\end{align*}
The item will cost her \$8.

Wendy also uses her formula to find her savings on a \$40 item by 
evaluating
\begin{align*}
  d(40)&=0.8\cdot 40\\
  &=32
\end{align*}
and concludes that she will save \$8 using her coupon.
\end{pccexample}

%===================================
%   Author: Hughes
%   Date:   November 2012
%===================================
\begin{pccexample}[Function evaluation]
Let $g$ be the piecewise-defined function that has formula  
\[
    g(x)=
    \begin{cases}
      -13, & x\leq -4\\
      2, & -4<x < 3\\
      7, &  x>3
    \end{cases}
\]
Evaluate each of the following
\begin{multicols}{5}
  \begin{enumerate}
    \item $g(-5)$
    \item $g(-4)$
    \item $g(0)$
    \item $g(3)$
    \item $g(53)$
  \end{enumerate}
\end{multicols}
\begin{pccsolution}
  \begin{enumerate}
    \item To evaluate $g(-5)$ we first need to identify which part of the 
      domain is appropriate. Since $-5\leq -4$, we use the formula in the 
      \emph{first} row of $g(x)$, and therefore
      \[
            g(-5)=-13
      \]
    \item Since $-4\leq -4$, we use the \emph{first} row in the formula for $g(x)$ again, so
      \[
            g(-4)=-13
      \]
    \item Since $-4<0<3$ we need to use the \emph{second} row in the formula for $g(x)$, so
      \[
            g(0)=2
      \]
    \item To evaluate $g(3)$ we need to find the appropriate interval in the formula
      for $g(x)$. Notice that $3$ does not fall into any of the intervals! This means
      that $g(3)$ is undefined.
    \item We note that $53>3$, so we need to use the \emph{third} row of the formula for $g(x)$, so
      \[
            g(53)=7
      \]
  \end{enumerate}
\end{pccsolution}
\end{pccexample}

%===================================
%   Author: Hughes
%   Date:   November 2012
%===================================
\begin{pccexample}
  \fixthis{more complictaed piecewise formula using radicals, quadratics etc} 
  \[
        f(t)=
        \begin{cases}
          t^2, & t<-3\\
          4-5t, & -3\leq t< 6\\
          \sqrt{t} & t>6
        \end{cases}
  \]
\end{pccexample}

\begin{exercises}
%===================================
%   Author: Hughes
%   Date:   October 2012
%===================================
\begin{problem}[Find a formula from a graph]
Consider the functions $F$, $G$, $H$, and $J$ that have been graped in 
\cref{fun:fig:piecewise}. Find a formula for 
each function.
\begin{multicols}{4}
	\begin{subproblem}
		$F$  
		\begin{shortsolution}
			$
			F(x)=
			\begin{cases}
				1,  & -2<x<-1 \\
				2,  & -1<x<0  \\
				-1, & 0<x<1   \\
				-2, & 1<x<2   
			\end{cases}
			$
		\end{shortsolution}
	\end{subproblem}
	\begin{subproblem}
		$G$
		\begin{shortsolution}
			$
			G(x)=
			\begin{cases}
				-1, & -2<x<-1 \\
				-2, & -1<x<0  \\
				1,  & 0<x<1   \\
				-1, & 1<x<2   
			\end{cases}
			$
		\end{shortsolution}
	\end{subproblem}
	\begin{subproblem}
		$H$
		\begin{shortsolution}
			$
			H(x)=
			\begin{cases}
				-1, & -2<x<-1 \\
				0,  & -1<x<0  \\
				1,  & 0<x<1   \\
				2,  & 1<x<2   
			\end{cases}
			$
		\end{shortsolution}
	\end{subproblem}
	\begin{subproblem}
		$J$
		\begin{shortsolution}
			$
			J(x)=
			\begin{cases}
				2,  & -2<x<-1 \\
				1,  & -1<x<0  \\
				0,  & 0<x<1   \\
				-2, & 1<x<2   
			\end{cases}
			$
		\end{shortsolution}
	\end{subproblem}
\end{multicols}

\begin{figure}[!htb]
	\begin{widepage}
	\setlength{\figurewidth}{.2\textwidth}
	\centering
	\begin{subfigure}{\figurewidth}
		\begin{tikzpicture}
			\begin{axis}[
				framed,
				xmin=-3,xmax=3,
				ymin=-3,ymax=3,
				xtick={-2,0,...,2},
				ytick={-2,0,...,2},
				minor xtick={-1,1},
				minor ytick={-1,1},
				grid=both,
				]
				\addplot[pccplot,-] expression[domain=-2:-1]{1};
				\addplot[pccplot,-] expression[domain=-1:0]{2};
				\addplot[pccplot,-] expression[domain=0:1]{-1};
				\addplot[pccplot,-] expression[domain=1:2]{-2};
				\addplot[holdot]coordinates{(-2,1)(-1,1)(-1,2)(0,2)(0,-1)(1,-1)(1,-2)(2,-2)};
			\end{axis}
		\end{tikzpicture}
		\caption{$y=F(x)$}
		\label{fun:fig:piecewise1}
	\end{subfigure}
	\hfill
	\begin{subfigure}{\figurewidth}
		\begin{tikzpicture}
			\begin{axis}[
				framed,
				xmin=-3,xmax=3,
				ymin=-3,ymax=3,
				xtick={-2,0,...,2},
				ytick={-2,0,...,2},
				minor xtick={-1,1},
				minor ytick={-1,1},
				grid=both,
				]
				\addplot[pccplot,-] expression[domain=-2:-1]{-1};
				\addplot[pccplot,-] expression[domain=-1:0]{-2};
				\addplot[pccplot,-] expression[domain=0:1]{1};
				\addplot[pccplot,-] expression[domain=1:2]{-1};
				\addplot[holdot]coordinates{(-2,-1)(-1,-1)(-1,-2)(0,-2)(0,1)(1,1)(1,-1)(2,-1)};
			\end{axis}
		\end{tikzpicture}
		\caption{$y=G(x)$}
		\label{fun:fig:piecewise2}
	\end{subfigure}
	\hfill
	\begin{subfigure}{\figurewidth}
		\begin{tikzpicture}
			\begin{axis}[
				framed,
				xmin=-3,xmax=3,
				ymin=-3,ymax=3,
				xtick={-2,0,...,2},
				ytick={-2,0,...,2},
				minor xtick={-1,1},
				minor ytick={-1,1},
				grid=both,
				]
				\addplot[pccplot,-] expression[domain=-2:-1]{-1};
				\addplot[pccplot,-] expression[domain=-1:0]{0};
				\addplot[pccplot,-] expression[domain=0:1]{1};
				\addplot[pccplot,-] expression[domain=1:2]{2};
				\addplot[holdot]coordinates{(-2,-1)(-1,0)(-1,-1)(0,0)(0,1)(1,1)(1,2)(2,2)};
			\end{axis}
		\end{tikzpicture}
		\caption{$y=H(x)$}
		\label{fun:fig:piecewise3}
	\end{subfigure}
	\hfill
	\begin{subfigure}{\figurewidth}
		\begin{tikzpicture}
			\begin{axis}[
				framed,
				xmin=-3,xmax=3,
				ymin=-3,ymax=3,
				xtick={-2,0,...,2},
				ytick={-2,0,...,2},
				minor xtick={-1,1},
				minor ytick={-1,1},
				grid=both,
				]
				\addplot[pccplot,-] expression[domain=-2:-1]{2};
				\addplot[pccplot,-] expression[domain=-1:0]{1};
				\addplot[pccplot,-] expression[domain=0:1]{0};
				\addplot[pccplot,-] expression[domain=1:2]{-2};
				\addplot[holdot]coordinates{(-2,2)(-1,2)(-1,1)(0,1)(0,0)(1,0)(1,-2)(2,-2)};
			\end{axis}
		\end{tikzpicture}
		\caption{$y=J(x)$}
		\label{fun:fig:piecewise4}
	\end{subfigure}
	\caption{}
	\label{fun:fig:piecewise}
	\end{widepage}
\end{figure}
\end{problem}
\end{exercises}